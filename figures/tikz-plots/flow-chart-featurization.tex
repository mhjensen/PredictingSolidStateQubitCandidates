\setlength{\abovecaptionskip}{10cm}
\begin{figure}[!ht]
\begin{picture}(20,20)
\setlength{\unitlength}{0.17in}

%\put(13.5, 0.5){\makebox{\thead{\textbf{Matminer featurizer}}}}
%\put(0.,-22){\framebox(33.5,22){\thead{}}}

\put(-16,-8.5){\vector(1,0){2.5}}
\put(-13.5,-10.5){\framebox(4,4){\thead{Partition \\ dataset}}}
\put(-9.5,-8.5){\vector(3,0){2}}
\put(-7.5,-10.5){\framebox(5,4){\thead{Get objects \\ from MP}}}
\put(-2.5,-8.5){\line(3,0){1}}
\put(-1.5,-8.5){\line(0,1){6}}
\put(-1.5,-8.5){\line(0,-1){6}}

\put(-1.5,-2.5){\vector(3,0){1}}
\put(-0.5,-3.5){\framebox(9,2){Featurize composition}}
\put(8.5,-2.5){\line(3,0){1}}

\put(-1.5,-5.5){\vector(3,0){1}}
\put(-0.5,-6.5){\framebox(9,2){Featurize structure}}
\put(8.5,-5.5){\line(3,0){1}}

\put(-1.5,-8.5){\vector(3,0){1}}
\put(-0.5,-9.5){\framebox(9,2){Featurize site}}
\put(8.5,-8.5){\line(3,0){1}}

\put(-1.5,-11.5){\vector(3,0){1}}
\put(-0.5,-12.5){\framebox(9,2){Featurize dos}}
\put(8.5,-11.5){\line(3,0){1}}

\put(-1.5,-14.5){\vector(3,0){1}}
\put(-0.5,-15.5){\framebox(9,2){Featurize band}}
\put(8.5,-14.5){\line(3,0){1}}

\put(9.5,-8.5){\line(0,1){6}}
\put(9.5,-8.5){\line(0,-1){6}}

\put(9.5,-8.5){\vector(2,0){1}}
\put(10.5,-10.5){\framebox(5.5,4){\thead{Done with \\ all partitions?}}}

\put(14, -12.){\makebox{NO}}
\put(13.5,-10.5){\line(0,-1){8}}
\put(13.5,-18.5){\vector(-1,0){16}}
\put(-7.5,-20.5){\framebox(5,4){\thead{Select next \\ partition}}}
\put(-5,-16.5){\vector(0,1){6}}


\put(16, -8.5){\vector(1,0){3.5}}
\put(16.5, -8.){\makebox{YES}}

\end{picture}
\caption{The process of the matminer featurizer step.  
%as seen in Figure \ref{fig:flowchart-makedata}. 
To limit the memory and computational usage, the data is partitioned into smaller subsets where the respective pymatgen objects are obtained through a query to be used in the following featurization steps. This process is repeated iteratively until all the data has been featurized. Abbreviations used are Materials Project (MP), density of states (DOS) and electronic band structure (band).}
\label{fig:flowchart-featurization}
\end{figure}
%\vskip15cm
%\setlength{\abovecaptionskip}{10cm}
